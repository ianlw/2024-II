\documentclass[a4paper]{article}
\usepackage[utf8]{inputenc}
\usepackage{graphicx}
\usepackage{geometry}
\usepackage{float}
\usepackage{afterpage}
\usepackage{listings}
\usepackage{xcolor}

\lstdefinestyle{mystyle}{
    language=Python,
    basicstyle=\small\ttfamily,
    keywordstyle=\color{blue},
    commentstyle=\color{gray},
    stringstyle=\color{green},
    breaklines=true,
    showstringspaces=false,
    numbers=left,
    numberstyle=\tiny,
    numbersep=5pt,
    frame=tb,
    framexleftmargin=16pt,
    framexrightmargin=0pt,
    xleftmargin=16pt,
    belowskip=10pt,
    aboveskip=10pt
}

% Configuración de los márgenes
\geometry{
    left=2.5cm,
    right=2.5cm,
    top=2.5cm,
    bottom=2.5cm,
}


\begin{document}
\newgeometry{left=3cm,right=3cm,top=2cm,bottom=2cm}
\begin{titlepage}

%--------------- Nuevo comendo de linea ----------------->
\newcommand{\linea}{\rule{\linewidth}{0.7mm}} 
\center
%--------------- Universidad, facultad y carrera ----------------->
\textbf{\Large UNIVERSIDAD NACIONAL DE SAN ANTONIO ABAD DEL CUSCO}\\[0.2cm]
\textbf{\Large FACULTAD DE INGENIERÍA ELÉCTRICA, ELECTRÓNICA,INFORMÁTICA Y MECÁNICA}\\[0.2cm]
\textbf{\Large INGENIERÍA INFORMÁTICA Y DE SISTEMAS\\[0.6cm]}

%--------------- Escudos png ----------------->
\includegraphics[width=8cm]{escudo-unsaac.png}
\vfill

%--------------- Tema ----------------->
\linea
\\[0.3cm]
% \vfill
\textbf{\LARGE Laboratorio 3 - Rabin-Karp-Matcher}\\[0.2cm]
\linea \\
\vfill

%--------------- Integrantes ----------------->
\textbf{\Large Integrantes:}\\[0.2cm]
%Integrantes del grupo
    {\large Huacho Cruz David Ali }\\
    \textit{211855}\\[0.1cm]

    {\large Huisa Nina Yimy Yohel }\\
    \textit{211857}\\[0.1cm]

    {\large Gutierrez Alfaro Roland Einar}\\
    \textit{210929}\\[0.1cm]

    {\large Quispe Ventura Ian Logan Will}\\
    \textit{211359}\\[0.1cm]

    {\large Rodriguez Pauccara Cristian Diego }\\
    \textit{210942}\\[0.1cm]
    % \vfill

%--------------- Profesor y curso ----------------->
\vspace{0.1cm}
    \textit{\Large Docente:}\\
    \textbf{\large Lauro Enciso Rodas}\\
\vspace{0.5cm}
    \textit{\Large Curso:}\\
    \textbf{\large Algoritmos Avanzados}\\
    \vfill

\vspace{0.5cm}
\textbf{\Large Cusco - Perú }\\
    \textbf{\large 2024}\\
    \newpage
    \end{titlepage}

\restoregeometry
\newpage
\section{Guía de usuario}
Ejecutar el archivo con extención .exe en windows, no es necesario instalar ninguna dependencia.

\subsection{Uso básico}
Ejecutar el programa haciendo doble click en el archivo 'Rabin-Karp-Matcher.exe' o también desde la consola de Windows (cmd), posicionandonos en la carpeta donde se encuentra el programa, con ayuda del comando cd (change directory), para ejecutar el programa simplementes escribiremos ./Rabin-Karp-Matcher.exe.

\subsection{Características principales}
\begin{itemize}
    \item La aplicación implementa el algoritmo de Rabin-Karp-Matcher que permite encontrar patrones dentro de cadenas. 
    \item La aplicación està desarrollada en Visual Studio usando c++ como lenguaje de programaciòn. 
\end{itemize}

\subsection{Ejemplo de uso}
\begin{enumerate}
    \item Completa los campos propocionados en la interfáz gráfica, que son:
    \begin{itemize}
        \item \textbf{T:} El string donde se buscarà el patròn .
        \item \textbf{P:} El patròn a buscar.
        \item \textbf{d:} El nùmero base del sistema (puede ser el tamaño del alfabeto, 256 para ASCII o 10 para numeros en esta base).
        \item \textbf{q:} Un nùmero primo para el mòdulo.
    \end{itemize}
    \item Haga clic en "Calcular" para ejecutar el algortimo de Rabin-Karp-Matcher.
\end{enumerate}
 Inmediatamente se mostrarà el resultado indicando la posiciòn donde se encontrò el patròn. 
\end{document}


